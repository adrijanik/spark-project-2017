%%%%%%%%%%%%%%%%%%%%%%%%%%%%%%%%%%%%%%%%%%%%%%%%%%%%%%%%%%%%%%%%%%%%%%
% LaTeX Example: Project Report
%
% Source: http://www.howtotex.com
%
% Feel free to distribute this example, but please keep the referral
% to howtotex.com
% Date: March 2011 
% 
%%%%%%%%%%%%%%%%%%%%%%%%%%%%%%%%%%%%%%%%%%%%%%%%%%%%%%%%%%%%%%%%%%%%%%
% How to use writeLaTeX: 
%
% You edit the source code here on the left, and the preview on the
% right shows you the result within a few seconds.
%
% Bookmark this page and share the URL with your co-authors. They can
% edit at the same time!
%
% You can upload figures, bibliographies, custom classes and
% styles using the files menu.
%
% If you're new to LaTeX, the wikibook is a great place to start:
% http://en.wikibooks.org/wiki/LaTeX
%
%%%%%%%%%%%%%%%%%%%%%%%%%%%%%%%%%%%%%%%%%%%%%%%%%%%%%%%%%%%%%%%%%%%%%%
% Edit the title below to update the display in My Documents
%\title{Project Report}
%
%%% Preamble
\documentclass[paper=a4, fontsize=11pt]{scrartcl}
\usepackage[T1]{fontenc}
\usepackage{fourier}

\usepackage[english]{babel}															% English language/hyphenation
\usepackage[protrusion=true,expansion=true]{microtype}	
\usepackage{amsmath,amsfonts,amsthm} % Math packages
\usepackage[pdftex]{graphicx}	
\usepackage{url}


%%% Custom sectioning
\usepackage{sectsty}
\allsectionsfont{\centering \normalfont\scshape}


%%% Custom headers/footers (fancyhdr package)
\usepackage{fancyhdr}
\pagestyle{fancyplain}
\fancyhead{}											% No page header
\fancyfoot[L]{}											% Empty 
\fancyfoot[C]{}											% Empty
\fancyfoot[R]{\thepage}									% Pagenumbering
\renewcommand{\headrulewidth}{0pt}			% Remove header underlines
\renewcommand{\footrulewidth}{0pt}				% Remove footer underlines
\setlength{\headheight}{13.6pt}


%%% Equation and float numbering
\numberwithin{equation}{section}		% Equationnumbering: section.eq#
\numberwithin{figure}{section}			% Figurenumbering: section.fig#
\numberwithin{table}{section}				% Tablenumbering: section.tab#


%%% Maketitle metadata
\newcommand{\horrule}[1]{\rule{\linewidth}{#1}} 	% Horizontal rule

\title{
		%\vspace{-1in} 	
		\usefont{OT1}{bch}{b}{n}
		\normalfont \normalsize \textsc{University of Nice, Sophia Antipolis} \\ [25pt]
		\horrule{0.5pt} \\[0.4cm]
		\huge Project in Spark 2017 \\
		\horrule{2pt} \\[0.5cm]
}
\author{
		\normalfont \normalsize
        Adrianna Janik\\	\normalfont \normalsize
Ion Mosnoi\\	\normalfont \normalsize
Lei Guo \\ \normalsize
        \today
}
\date{}


%%% Begin document
\usepackage{listings}
\begin{document}
\maketitle
\section{Task}
Firstly we uncompressed the data stored in ling-spam.zip folder with \textit{Extract all} command. 
Secondly we open Virtual Box machine with Hortonworks, we signed in with maria\_dev username and maria\_dev password on Ambari available under 127.0.0.1:8080 ip address. We have selected \textit{Files view}, than navigated to \textit{/tmp} folder and created directories \textit{tmp/ling-spam/ham} and \textit{ling-spam/spam}. Following that we logged in with ssh credentials to Hortonworks machine
\begin{lstlisting}[language=bash]
$ssh root@127.0.0.1 -p 2222
\end{lstlisting}
In the meantime upload to the virtual machine ling-spam.zip with:
\begin{lstlisting}[language=bash]
$sudo scp -P 2222 ../ling-spam.zip  root@127.0.0.1:/tmp/
\end{lstlisting}
We unzipped ling-spam.zip with:
\begin{lstlisting}[language=bash]
$unzip ling-spam.zip -d /tmp/ling-spam
\end{lstlisting}
We putted files into /tmp/ling-spam/ folder in hdfs with:
\begin{lstlisting}[language=bash]
$hdfs dfs -put ./ling-spam/ham /tmp/ling-spam/ham
$hdfs dfs -put ./ling-spam/spam /tmp/ling-spam/spam
\end{lstlisting}

\section{Task}
Installation of sbt:
\begin{lstlisting}[language=bash]
$wget http://dl.bintray.com/sbt/rpm/sbt-0.13.12.rpm
\end{lstlisting}
Edit file /etc/yum.repos.d/sandbox.repo:
\begin{lstlisting}[language=bash]
~[sandbox]
~name=Sandbox repository (tutorials)
~gpgcheck=0
~enabled=0
~baseurl=http://dev2.hortonworks.com.s3.amazonaws.com/repo/dev/master/utils/
\end{lstlisting}

\begin{lstlisting}[language=bash]
$yum clean all
$yum update
$sudo yum localinstall sbt-0.13.12.rpm
$sbt -update
$sudo scp -P 2222 -r ../spamTopWords/*  root@127.0.0.1:/tmp/spamTopWords/
$sbt package
\end{lstlisting}



\section{Task}
\section{Task}
\section{Task}

Lorem ipsum dolor sit amet, consectetuer adipiscing elit. Aenean commodo ligula eget dolor. Aenean massa. Cum sociis natoque penatibus et magnis dis parturient montes, nascetur ridiculus mus. Donec quam felis, ultricies nec, pellentesque eu, pretium quis, sem. In enim justo, rhoncus ut, imperdiet a, venenatis vitae, justo. Nullam dictum felis eu pede mollis pretium. Integer tincidunt. Cras dapibus. Vivamus elementum semper nisi. Aliquam lorem ante, dapibus in, viverra quis, feugiat a, tellus:
\begin{align} 
	\begin{split}
	(x+y)^3 	&= (x+y)^2(x+y)\\
					&=(x^2+2xy+y^2)(x+y)\\
					&=(x^3+2x^2y+xy^2) + (x^2y+2xy^2+y^3)\\
					&=x^3+3x^2y+3xy^2+y^3
	\end{split}					
\end{align}
Phasellus viverra nulla ut metus varius laoreet. Quisque rutrum. Aenean imperdiet. Etiam ultricies nisi vel augue. Curabitur ullamcorper ultricies 

\subsection{Heading on level 2 (subsection)}
Lorem ipsum dolor sit amet, consectetuer adipiscing elit. 
\begin{align}
	A = 
	\begin{bmatrix}
	A_{11} & A_{21} \\
  	A_{21} & A_{22}
	\end{bmatrix}
\end{align}
Aenean commodo ligula eget dolor. Aenean massa. Cum sociis natoque penatibus et magnis dis parturient montes, nascetur ridiculus mus. Donec quam felis, ultricies nec, pellentesque eu, pretium quis, sem.

\subsubsection{Heading on level 3 (subsubsection)}
Nulla consequat massa quis enim. Donec pede justo, fringilla vel, aliquet nec, vulputate eget, arcu. In enim justo, rhoncus ut, imperdiet a, venenatis vitae, justo. Nullam dictum felis eu pede mollis pretium. Integer tincidunt. Cras dapibus. Vivamus elementum semper nisi. Aenean vulputate eleifend tellus. Aenean leo ligula, porttitor eu, consequat vitae, eleifend ac, enim.

\paragraph{Heading on level 4 (paragraph)}
Lorem ipsum dolor sit amet, consectetuer adipiscing elit. Aenean commodo ligula eget dolor. Aenean massa. Cum sociis natoque penatibus et magnis dis parturient montes, nascetur ridiculus mus. Donec quam felis, ultricies nec, pellentesque eu, pretium quis, sem. Nulla consequat massa quis enim. 


\section{Lists}

\subsection{Example for list (3*itemize)}
\begin{itemize}
	\item First item in a list 
		\begin{itemize}
		\item First item in a list 
			\begin{itemize}
			\item First item in a list 
			\item Second item in a list 
			\end{itemize}
		\item Second item in a list 
		\end{itemize}
	\item Second item in a list 
\end{itemize}

\subsection{Example for list (enumerate)}
\begin{enumerate}
	\item First item in a list 
	\item Second item in a list 
	\item Third item in a list
\end{enumerate}
%%% End document
\end{document}
